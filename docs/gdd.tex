\documentclass[11pt]{article}

\usepackage{listings}
\usepackage{textcomp}

\lstset{}

\addtolength{\oddsidemargin}{-.925in}
\addtolength{\evensidemargin}{.925in}
\addtolength{\textwidth}{1.45in}
\addtolength{\topmargin}{-.875in}
\addtolength{\textheight}{1.75in}

\title{The Devil's Fingers}
\author{A game by John Torsten}

\begin{document}
	\maketitle
	\tableofcontents
	\clearpage

	\section{Terms}
	
	\begin{itemize}
		\item \textbf{Locale}: A distinct physical location in the game world, such as a room or a clearing in the woods.
		\item \textbf{PC}: abbreviation for \textit{player character}, referring to the player of the game.
		\item \textbf{Region}: An area in the game world with distinct environmental characteristics consisting of several locales.
		\item \textbf{TDF}: abbreviation for \textit{The Devil's Fingers}
	\end{itemize}
	
	\section{Concept}
	
	\textit{The Devil's Fingers} (abbr. \textit{TDF}) is a text-based game about finding a way to escape an unbelievably hostile forest tucked in the desolate corners of Caledonia County, Vermont in the year of 1987.
	
	\textit{TDF}'s core gameplay revolves around exploring the world, interacting with the environment, and fighting enemies by typing textual commands into a console. Unlike most classic text-based adventure games, however, \textit{TDF} runs in real-time and does not wait for the player's input.
	
	\section{Themes}
	
	The game revolves around a few central themes:
	
	\begin{itemize}
		\item Power seduces.
		\item Negligence destroys.
		\item Persistence is a necessity.
	\end{itemize}
	
	\section{Tone}
	
	\textit{TDF}'s tone is gothic, macabre, and unsettling. The forest hosts the remnants of an uncanny civilization that has long since fallen to supernatural disarray, its existence hidden from locals due to the steep mountains and treacherous terrain that characterizes the environment. The remaining living inhabitants of the area are violent creatures of paranormal and mythical demeanor. Nothing is forgiving, and many things are lethal.
	
	\section{Mechanics}
	
	\subsection{Exploration Mode}
	
	The majority of gameplay takes place in \textbf{Exploration mode}. In this mode, the player is able to traverse the map, examine paths and features, interact with things, and regulate their vital metrics.
	
	\textbf{Exploration mode} is split into four distinct views discussed later in the UI section.
	
	\begin{itemize}
		\item The \textbf{Console View} contains the command console with which the player enters commands and a scrollable section with the textual output of the game.
		\item The \textbf{Inventory View} is designed for interacting with items in the player's inventory and the environment.
		\item The \textbf{Map View} provides an overhead map to ease the player's navigation through the treacherous woods.
		\item The \textbf{Information View} provides a consolidated view of important information about the player's state, including vital statistics and any active modifiers or induced effects.
	\end{itemize}

	\textbf{Using Commands}
	\newline
	
	To perform actions, the player types commands into the console, such as "go north", which would cause the player to attempt to travel to the locale directly to the north. The resulting output of an action is printed to the console pane in the \textbf{Console View}.
	
	Most commands have several aliases; typing "go n" or even just "north" would accomplish the same action in this case.
	
	All player actions can be performed in the \textbf{Console View} by typing commands into the console, however the three other views provide alternative interfaces that ease interaction with items and the environment. Commands are discussed more thoroughly in the \textbf{Commands} section.\linebreak
	
	\textbf{Navigation}
	\newline
	
	The game's map can be split into several \textbf{regions}, which are each a collection of connected \textbf{locales}. Locales can contain exits in twelve possible directions: the eight cardinal directions, up, down, inside, and outside.
	
	The player can use the EXAMINE command to receive textual output regarding which exits are present in their current locale. The player can use the GO command to attempt to navigate in a direction.
	
	\subsection{Vitality}
	
	The player has four vital metrics that they must take action to regulate or die. These metrics are:
	
	\begin{itemize}
		\item \textit{Health}: A measure of physical vitality. 100 represents being in ideal physical state while 0 represents death.
		\item \textit{Sanity}: A measure of mental stature. 100 represents a calm, sane mental state while 0 represents a neurotic insanity -- a complete inability to distinguish hallucinations from reality.
		\item \textit{Energy}: A measure of physical energy. 100 represents being energetic while 0 represents complete exhaustion and involuntary immobility.
		\item \textit{Temperature}: A measure of body temperature. Unlike the previous metrics which are quantified on a scale of 0 to 100, this is a standard temperature readout in Fahrenheit and Celsius representing the main character's internal body temperature. 37\textdegree C (98.6\textdegree F) is considered an ideal body temperature. Dropping below this temperature results in freezing while exceeding this temperature results in overheating. Both yield negative effects, eventually resulting in death.
	\end{itemize}

	These vital metrics can be regulated through the use of items, such as applying a bandage to stop bleeding or consuming food to replenish energy. Items that can be equipped can also apply modifiers that affect vital metrics, such as wearing a coat to conserve body heat or adorning a charm that decreases the intensity of sanity-draining effects.
	
	\subsection{Combat Mode}
	
	When the PC enters a locale occupied by an enemy or an enemy enters the locale occupied by the PC, the PC exits \textbf{Exploration mode} and enters \textbf{Combat mode}. The PC will stay in this mode until all enemies in the locale are subdued or the PC dies. If multiple enemies are present in the PC's current locale, the PC must subdue all of them, one by one, while the enemies can "tag out" for each other.
	
	\textbf{Combat mode} has a separate user interface from \textbf{Exploration mode} that includes sprites for the PC and the active enemy, bars representing their health and cooldown statuses, a text area for narrating the moves being performed, and a text input for the player to enter commands that perform moves.
	
	\subsection{Combat Moves}
	During combat, the PC and the current active enemy perform \textit{combat moves}. The PC performs combat moves by typing commands into a text input on the user interface, while the active enemy performs combat moves automatically when it is not on cooldown.
	
	Each combat move has a distinct \textit{cooldown} during which the acting entity cannot perform other moves. This cooldown is depicted in the user interface through a blue bar that depletes upon using a move and refills fully by the time the cooldown is no longer active. Some moves are attacks, such as a wolf biting or the PC swinging a wrench. Moves can also be other actions that are performed during combat, such as blocking, dodging, or retreating.
	
	When performed, a move can produce an \textit{instant effect}, \textit{active effect}, or both. Instant effects occur instantly after a move is performed, while active effects continue to occur until the move's cooldown is fully depleted. An example of an instant effect would be the blunt damage dealt by the PC bashing an enemy with a shovel. On the other hand, the "dodge" move yields no instant effect, but has an active effect of increasing the chance the active enemy misses their attack for the entire duration of the move's cooldown.
	
	Combat moves can be classified as either \textit{attack} or \textit{maneuver}. Attacks yield instant damage and set the PC into cooldown, encapsulating moves such as punching, kicking, or headbutting. Maneuvers do not necessarily yield immediate damage, but instead produce a special effect for the duration of the cooldown, including moves like dodging, blocking, or grappling.
	
	\textbf{Attacks}
	
	\begin{itemize}
		\item \textit{Kick}: 4-8 base damage, 3s cooldown.
	\end{itemize}
	
	\section{Commands}
	While in \textbf{Exploration mode}, the player's main interface with the game world is via entering commands into the console. Each command has several aliases that can be inputted instead of the command name. Some commands require a certain syntax; the syntaxes provided below note required arguments with brackets and optional arguments with parentheses.
	\subsection{ATTACK}
	\textbf{Aliases}: \textit{assault, bonk, fight, hit, kick, punch}\newline
	\textbf{Syntax}: \textit{attack [target]}\newline
	The PC attempts to use physical violence on the specified target. If a melee weapon is equipped, the PC will attempt to use that, otherwise a bare-fisted punch is attempted. Attacking a sentient entity may result in initiating \textbf{Combat mode}.
	\subsection{EAT}
	\textbf{Aliases}: \textit{bite, chew, consume, devour, ingest, inhale, masticate, nosh, swallow}\newline
	\textbf{Syntax}: \textit{eat [target]}\newline
	The PC attempts to eat the specified target if it is edible. Most things are inedible, and some things are catastrophically harmful if eaten.
	\subsection{EQUIP}
	\textbf{Aliases}: \textit{adorn, arm, don, dress, endow, fit, furnish, gear, put on, wear}\newline
	\textbf{Syntax}: \textit{equip [target]}\newline
	If the specified target is equipable, the PC attempts to equip it. Anything wielded in the hands or worn on the body needs to be equipped in order to be used effectively.
	\subsection{EXAMINE}
	\textbf{Aliases}: \textit{gander, look, perceive, peruse, search, where, whereami}\newline
	\textbf{Syntax}: \textit{examine (target)}\newline
	The PC looks attentively at the specified target and outputs his notable examinations, if any, to the console view. If no target is specified, the PC examines their general surroundings and reports anything notable.
	\subsection{GO}
	\textbf{Aliases}: \textit{cross, leave, migrate, move, proceed, progress, relocate, travel, walk}\newline
	\textbf{Syntax}: \textit{go [direction]}\newline
	The PC attempts to navigate from their current locale to another in the given direction. If travel is impossible, the reason why is printed to the console view. If travel is possible in the specified direction, the PC will begin traveling through their current locale to their destination. The amount of time it takes to travel is dependent upon environmental variables such as terrain, visibility, and the PC's vitality metrics.
	\subsection{HIDE}
	\textbf{Aliases}: \textit{camouflage, cloak, cover, disguise, duck, obscure, shadow, shelter, shroud,  tuck}\newline
	\textbf{Syntax}: \textit{hide (around/among/amongst/behind/in/over/under feature)}\newline
	The PC attempts to hide their self within their current locale or obscured by the specified feature within the locale.
	\subsection{THROW}
	\textbf{Aliases}: \textit{catapult, chuck, discharge, fire, flick, fling, heave, hurl, launch, lob, pitch, send, sling, toss, volley, yeet}\newline
	\textbf{Syntax}: \textit{throw [object] (at [target])}\newline
	If the specified object is throwable, the PC attempts to throw it at the specified target. If no target is specified, the PC throws the object on the ground of their current locale.
	\section{Combat Moves}
	In \textbf{Combat mode}, normal commands are not available, and instead the player inputs \textit{moves} in a similar fashion. Each move entails a certain amount of \textit{cooldown} during which the PC is unable to perform another move. Moves have varying affects from dealing damage to attempting escape.
	
	\section{Game Entities}
	\subsection{Locales}

	\begin{lstlisting}
interface Locale {
	name: string;
	display: string;
	comments: string[];
	containers: Container[];
	coordinates: number[];
	enemies: Enemy[];
	enterPhrase: string;
	examinePhrase: string;
	exitPhrase: string;
	exits: Exit[];
	features: string[];
	items: Item[];
	loot: Loot[];
	spawns: Spawn[];
	visits: number;
	temperature: number;
	visibility: number;
}
	\end{lstlisting}
	
	A locale is an arbitrarily-sized physical unit of space within the game. The player travels between these discrete locales to traverse the map. Each locale may contain containers, enemies, exits, features, and items. Locales can have exits in twelve different directions leading to other locales.
	
	\subsection{Regions}
	
	The game is separated into three regions which are each a collection of locales. Each region has its own physical characteristics as well as its own boss.
	
	The \textit{forest} is the region in which player starts. This region is a stretch of boreal woods characterized by rough natural terrain with rapid elevation changes and dense, thorny underbrush. The enemies encountered in this area are predatory wildlife and mythical beasts.
	
	The \textit{mansion} is a lavish but decrepit estate built on the summit of a hill in the \textit{forest}. Fallen to disarray and rummaged over the years, the interior is a poor respite from the cold, adorned with ornamental furniture whose intricacies and integrity have withered to time and neglect. Aside from rats, nightmarish beings lurk in the shadows as the \textit{Hunter} prowls its premises.
	
	The \textit{cemetery} is the somber resting place for thousands of corpses, some graves simple and effaced, others tucked deep within mausoleums and crypts. Wrought iron fences and intricately carved statues divide the area, which teems with the festering, restless souls whose bodies have rotted below.
	
	\subsection{NPCs}
	
	\textbf{The Carpenter}
	\newline
	
	\textit{The Carpenter} is Tamara Pierson, another survivor who became lost in the forest a day before the player after her vehicle was similarly obstructed along the way to a remote job. She is persistent, crude, and adamant in her decisions and convictions.
	
	The player never encounters her; she interfaces through messages left on objects that the player stumbles upon. \textit{The Carpenter} is thoroughly determined to escape the forest, leaving practical messages like carved arrows pointing to useful resources or tips for fighting certain enemies.
	
	\textbf{The Gardener}
	\newline
	
	\textit{The Gardener} is Cassie Ewing, a survivor stranded in the forest after getting lost during a hiking trip the previous night. Similarly to \textit{The Carpenter}, her only interface with the player is through messages left throughout the forest.
	
	\subsection{Enemies}
	
	Enemies are antagonistic entities strewn throughout the game world in various locales. Encountering an enemy will initiate combat. Enemies are mostly distinct to the different \textit{regions} in the game. Each enemy has one of three \textit{threat levels} (dangerous, perilous, or fatal) indicating how deadly it is to fight.\newline
	
	\textbf{Forest}
	\newline
	\rule{\textwidth}{0.4pt}
	
	--\textit{Dangerous}--
	\begin{itemize}
		\item Wolf
		\item Hawk
		\item Centipede
		\item Wasp
		\item Vulture
		\item Fungus
	\end{itemize}

	--\textit{Perilous}--
	\begin{itemize}
		\item Hound
		\item Troll
		\item Worm
		\item Spider
		\item Hornet
		\item Moth
		\item Wolf-Man
	\end{itemize}

	--\textit{Fatal}--
	\begin{itemize}
		\item Mutt
		\item Panther
		\item Arachnid
		\item Bear
		\item Serpent
		\item Bug
		\item Horror
	\end{itemize}

	\textbf{Mansion}
	\newline
	\rule{\textwidth}{0.4pt}
	
	--\textit{Dangerous}--
	\begin{itemize}
		\item Rat
		\item Shadow
		\item Imp
	\end{itemize}
	
	--\textit{Perilous}--
	\begin{itemize}
		\item Gargoyle
		\item Shade
	\end{itemize}
	
	--\textit{Fatal}--
	\begin{itemize}
		\item Nightmare
		\item Abomination
		\item Torment
	\end{itemize}

	\textbf{Cemetery}
	\newline
	\rule{\textwidth}{0.4pt}
	
	--\textit{Dangerous}--
	\begin{itemize}
		\item Wraith
		\item Phantom
		\item Spectre
		\item Living Blade
		\item Ghoul
		\item Echo
	\end{itemize}
	
	--\textit{Perilous}--
	\begin{itemize}
		\item Nightwraith
		\item Tormented
		\item Weeper
		\item Devil
		\item Devourer
	\end{itemize}
	
	--\textit{Fatal}--
	\begin{itemize}
		\item Dreadwraith
		\item Fiend
		\item Demon
		\item Void
		\item Necromancer
		\item Seraphim
	\end{itemize}
	
	\subsection{Bosses}
	
	Each region has one boss that serves as a primary antagonistic force that keeps the player on their guard and on the move. Each boss moves around the game world independently of the player, either guarding an area or hunting the player.\newline
	
	The forest is inhabited by the \textbf{Hulking Horror}, known by the past locals as Kuolema, the deity that upholds the boundaries of Caledonia Forest. It was originally an angel, permanently disfigured throughout its centuries of benevolently containing the demonic presence in Caledonia Forest. Its sacrifices have led it to become diseased and delusional, unable to distinguish between good or evil. Its only remaining motivation is to uphold the scarred boundary of the forest, assuring no supernatural presences drift into the mortal world and cause irreparable damage.\newline
	
	The mansion is home to the \textbf{Hunter}: whatever is left of the poor Amaya Basma. Amaya is an embodiment of anxiety, even in her afterlife. Having lived through an unfortunate childhood, she learned to trust no one. She never understood the state of carefree happiness others seemed to possess in life, and secretly wished for a disaster to strike her entire life. She became very religious and recluse, anticipating the end of times and stockpiling a bunker near Caledonia Forest. Her only companions are her two hunting dogs: Lohi and Prophet.

	It wasn't until a fateful day, at the age of 46, that she traveled into the forest's bounds to hunt and forage, only to find the disaster she sought for. Luckily, she had ample hunting gear and a predisposition for surviving. Some of her struggles are documented in journal pages scattered around the mansion, but she gradually lost sentience as she harnessed supernatural practices to survive and recover from grievous injuries. Throughout her struggles, she lost Lohi, but she was able to keep Prophet alive through the same heinous means as herself. Now her only modus operandi is to defend the mansion and her supernaturally prolonged life as long as she can.\newline
	
	The graveyard houses the \textbf{Red Wraith}, an embodiment of incomprehensible torment. Unspeakable suffering was incurred to open the rift under Caledonia Forest. Hundreds of souls who have experienced hellish deaths haunt the graveyard with sheer hatred at the absolute worst that humanity can be.
	
	\subsection{Items}
	
	\subsection{Weapons}
	
	\section{Player Experience}
	\subsection{Core Gameplay Loop}
	
	The core activity loop consists of exploring the world, fighting increasingly dangerous enemies, and looting items. A typical mid-game scenario would be:
	
	\begin{itemize}
		\item The player enters a barren farm plot, reads a brief description of the area as well as a description of the wolf standing guard over a partially consumed corpse.
		\item The player engages in combat with the wolf. Over the course of a few seconds, the player swings a crowbar to deal blunt damage, the wolf retaliates with a bite that injures the player, the player swings but misses, the wolf growls, and the player lands a final blow that forces the wolf to retreat.
		\item Combat ends and the player loots the locale of its valuables.
		\item The player consumes some handwarmers to improve their decreasing body temperature.
		\item The player examines the features in the field and discovers a recently-sealed hole in the soil.
		\item The player uses a shovel to dig up an item that was buried which allows travel to a new location.
	\end{itemize}
	
	\subsection{First hour of play}
	
	\begin{itemize}
		\item\textbf{First minute}: Get intrigued by the introduction cinematic and learn how to type commands.
		\item\textbf{5 mins}: Navigate around the starting zone, collect rudimentary equipment, and survive the first combat encounter.
		\item\textbf{10 mins}: Find better gear and successfully subdue an enemy in combat.
		\item\textbf{30 mins}: Discover a new region.
		\item\textbf{1 hour}: Encounter a boss.
	\end{itemize}

	\subsection{Progression}
	
	The map is arranged into \textit{zones} constrained by obstacles that require items, abilities, or special knowledge, referred to as \textit{keys} to overcome. These terms are not conveyed to the player, and are used for design purposes.
	
	\textbf{Zone A: Center}
	\textit{Keys}
	
	\section{User Interface}
	
	\textit{TDF}'s graphical user interface consists of several distinct views. Some views share elements, but each is distinct in functionality.
	
	\subsection{Main Menu}
	
	\subsection{Options Menu}
	
	\subsection{Console View}
	
	\subsection{Item View}
	
	\subsection{Map View}
	
	\subsection{Info View}
	
	\subsection{Cinematic View}
	
	\subsection{Dialogue View}
	
	\subsection{Death View}
	
\end{document}